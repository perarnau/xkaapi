\documentclass[a4paper, 11pt]{article}

\usepackage{graphicx}
\usepackage{graphics}
\usepackage{verbatim}
\usepackage{listings}
\usepackage{color}

\begin{document}

\title{Report: hierarchical workstealing in XKAAPI}
\author{XKAAPI team}
\date{}

\maketitle

\newpage
\tableofcontents
\addtocontents{toc}{\protect\setcounter{tocdepth}{1}}

% \part{title}
% \section{title}
% \subsection{title}


% Introduction
%
\newpage
\section{Introduction}
\paragraph{}
TODO: describe the need for HWS


% Building and running
%
\newpage
\section{Enabling HWS in XKAAPI}

\subsection{Build configuration}
\paragraph{}
No new configuration has been added to the build system. However, the
runtime has to be compiled with \textit{hwloc} and \textit{numa} support
for HWS to be enabled:\\
\begin{small}
\begin{lstlisting}[frame=tb]
./configure --with-hwloc --with-numa
\end{lstlisting}
\end{small}

\subsection{Environment variables}
\paragraph{}
The steal request emission routine has to be specialized by setting the
\textit{KAAPI\_EMITSTEAL} environment variable to ``hws'':\\
\begin{small}
\begin{lstlisting}[frame=tb]
$> KAAPI_HWS_LEVEL=hws ./a.out
\end{lstlisting}
\end{small}

\paragraph{}
Specific memory heriarchy levels can be used by setting the
\textit{KAAPI\_HWS\_LEVELS} environment variable. It consists of a comma
separated list of one or more of the following values:
\begin{itemize}
  \item ALL: enables all the levels,
  \item NONE: disables all the levels,
  \item L3: enable the L3 cache level,
  \item NUMA: enable the numa level,
  \item SOCKET: enable the socket level,
  \item MACHINE: enable the machine level,
  \item FLAT: enable the flat level.
\end{itemize}
For instance:\\
\begin{small}
\begin{lstlisting}[frame=tb]
$> KAAPI_HWS_LEVEL=hws KAAPI_HWS_LEVELS=FLAT,NUMA ./a.out
\end{lstlisting}
\end{small}

\paragraph{}
Not setting this variable enables the NUMA, SOCKET, MACHINE and FLAT
memory levels.

% Implementation
%
\newpage
\section{HWS implementation in XKAAPI}

\subsection{Overview}
\paragraph{}
The \textit{src/hws} directory has been added to the runtime sourcecode. It implements:
\begin{itemize}
\item subsystem initialization (\textit{kaapi\_hws\_initialize.c}),
\item task pushing (\textit{kaapi\_hws\_pushtask.c}),
\item steal request emission (\textit{kaapi\_hws\_emitsteal.c}),
\item adaptive task handling (\textit{kaapi\_hws\_adaptive.c}),
\item initial splitter (\textit{kaapi\_hws\_splitter.c}),
\item workstealing queue implementation (\textit{kaapi\_ws\_queue\_xxx.c}),
\item related performance counters (\textit{kaapi\_hws\_counters.c}),
\item scheduler synchronization (\textit{kaapi\_hws\_sched\_sync.c}).
\end{itemize}

\subsection{Hierarchy construction}
\paragraph{}
TODO
\begin{figure}[!hb]
\centering
\includegraphics[keepaspectratio=true, scale=0.6]{../dia/impl/main.jpeg}
\caption{HWS data structure relations}
\label{hws_impl}
\end{figure}

\subsection{Steal request emission algorithm}
\paragraph{}
The steal request emission entrypoint is the kaapi\_hws\_emitsteal routine. It
is still in progress, the final implementation will depend on the performances
achieved during the benchmarking process. Currently, the algorithm is as follow:
\begin{enumerate}
\item try to pop from the local queue, stop upon success.
\item steal in each parent level.
\item during the parent level iteration, try to steal in each child level.
\item if child level stealing fails, try to steal in parent level leaves.
\item fail the unanswered requests.
\end{enumerate}

\subsection{Workstealing queue virtualization}
\paragraph{}
To provide better flexibility regarding workstealing decision, a new \textit{queue}
interface has been added. It contains the following methods:
\begin{itemize}
  \item push: push a task in the queue,
  \item pop: pop a task from the local queue,
  \item steal: reply to a set of steal requests.
\end{itemize}
Currently, there is only a basic static lifo implementation.

\subsection{Emisteal routine virtualization}
\paragraph{}
The \textit{emitsteal} routine has been virtualized. The
\textit{KAAPI\_EMITSTEAL} environment variable controls the
implementation used:
\begin{itemize}
\item hws: use the hierarhical workstealing implemented by the
\textit{kaapi\_hws\_sched\_emitsteal} routine,
\item any other value: use the default \textit{kaapi\_sched\_emitsteal} routine.
\end{itemize}

\subsection{Workstealing blocks}
\paragraph{}
TODO


% Benchmarks
%
\newpage
\section{Benchmarks}
\paragraph{}
TODO


\end{document}
